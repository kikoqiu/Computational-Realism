\documentclass[11pt]{article}

% --- PACKAGES ---
\usepackage[utf8]{inputenc}
\usepackage{amsmath}
\usepackage{amssymb}
\usepackage[a4paper, margin=1in]{geometry}

% --- TITLE AND AUTHOR ---
\title{\textbf{The Entropic Dynamics Mechanism of Quantum Entanglement}}
\author{Haifeng Qiu} 


\begin{document}

% --- Revision Declaration Page ---
\begin{titlepage}
\thispagestyle{empty} % Removes header and footer from this page
\begin{center}
\vspace{4mm}
\textbf{Declaration of Corrections for Version 2.0 of this Manuscript}
\vspace{4mm}
\end{center}

We must begin by candidly acknowledging that in the initial version of this paper, our core prediction regarding the dynamical properties of the cosmic background field $\Sigma(P)$ [i.e., the content of Chapters 4 and 5] was an \textbf{arbitrary error} based on an oversimplified classical physics picture.

The initial version erroneously predicted that the $\Sigma(P)$ field is a ``quasi-static'' field that changes extremely slowly [with a coherent time $\tau_c$ on the order of hours or days]. Upon review, we immediately recognized that this conclusion is logically \textbf{incompatible} with the results of the CHSH inequality, which have been verified by precise experiments. A quasi-static, superdeterministic model is mathematically incapable of reproducing the quantum correlations that exceed the classical Bell's inequality upper bound.

We hereby explicitly \textbf{retract} the core prediction from Version 1.0 that the $\Sigma(P)$ field is ``quasi-static,'' and have completely updated the relevant content in this version.

This error also offers us a profound insight: \textbf{how difficult it is for an internal observer to perceive the truth.} We believe that scientific progress relies on unrelenting self-reflection and revolution based on logic and facts.

\vspace{1cm}

We apologize for any confusion that Version 1.0 may have caused and sincerely invite you to review and \textbf{challenge} our theoretical discourse.
\vspace*{\fill}
\end{titlepage}

% --- Main body of the paper starts on a new page ---
\clearpage


\maketitle
\thispagestyle{empty} % No page number on the title page
\newpage
\setcounter{page}{1}


% --- CHAPTER 1 ---
\section{Introduction — The Scale Emergence of Physics}

\subsection{A Unified Starting Point: From Bits to the Structural Scales of the Universe}

The framework of ``Computational Realism'' that we construct has at its core the following assertion: The entirety of physical reality as we know it consists of effective dynamics emerging at multiple, distinct structural scales from a single, deterministically ruled \textbf{bit computation field}. We divide this into four scales:

\begin{enumerate}
    \item   \textbf{The Bit-Scale:} The most fundamental, discrete computational substrate of the universe. Its dynamics are governed by a unique \texttt{Rule}, manifesting as pseudo-random chaos, which is the ultimate origin of all physical phenomena.
    
    \item   \textbf{The Particle-Scale:} Stable, local patterns with specific \textbf{topological structures}, formed by the self-organization of bits. Its dynamics (topological dynamics) are responsible for explaining the generation mechanism of \textbf{fundamental particles} and their \textbf{strong, weak, and electromagnetic interactions}.
    
    \item   \textbf{The Microscopic Scale of Entropic Effects:} This is the manifestation of the chaos at the \textbf{Bit-Scale} as \textbf{high-frequency fluctuations} in microscopic regions of spacetime. Its dynamics are responsible for injecting definite, yet unknowable, initial conditions into the \textbf{creation of quantum events}.
    
    \item   \textbf{The Statistical Scale of Entropic Effects:} This is the \textbf{smooth, classical background entropy structure field} that emerges after the chaos of the \textbf{Bit-Scale} is \textbf{statistically averaged} over macroscopic spacetime regions. Its dynamics (entropic dynamics) lead to the effects of \textbf{gravity} and \textbf{quantum measurement}.
\end{enumerate}

\subsection{Positioning and Core Thesis of This Paper}

The complete theory of Computational Realism aims to provide a unified explanation for all the emergence mechanisms, from the Bit-Scale to all higher scales. \textbf{This paper, however, will focus on the third and fourth scales mentioned above—namely, the two scales of ``entropic effects''}—and, based on this, provide a unified mechanistic explanation for \textbf{gravity} and \textbf{quantum entanglement}.

For the clarity of the arguments in this paper, the terms \textbf{``microscopic''} and \textbf{``macroscopic''} used hereafter will be \textbf{specifically and locally} defined as follows:
\begin{itemize}
    \item   \textbf{``Microscopic'' in this paper} refers specifically to the \textbf{``Microscopic Scale of Entropic Effects,''} i.e., the entropy field fluctuations that couple with quantum creation.
    \item   \textbf{``Macroscopic'' in this paper} refers specifically to the \textbf{``Statistical Scale of Entropic Effects,''} i.e., the classical entropy structure field that serves as the background for quantum measurement.
\end{itemize}

\noindent Based on this qualification, the core thesis of this paper is: The mystery of quantum entanglement can be fully understood as the inevitable deterministic consequence of the interaction between a \textbf{particle (a low-entropy structure emerging at the ``Particle-Scale'')} and the universal entropy field at its \textbf{``Microscopic Scale of Entropic Effects'' (influencing its creation)} and its \textbf{``Statistical Scale of Entropic Effects'' (influencing its measurement)}.


% --- CHAPTER 2 ---
\section{The Ontology of Entropy — A Multi-Scale Effective Field}

\subsection{Fundamental Postulate: Computational Entropy as the Sole Reality}
The cornerstone of our theory is that the ultimate measure of reality is \textbf{Computational Entropy (Kolmogorov Complexity, $S_C$)}. It measures the incompressibility of a pattern.

\subsection{Application of the Renormalization Idea: From the Fundamental Field to the Effective Field}
In order to study a discrete, fine-grained field defined at the bit-scale using continuous mathematics, we must define effective scales. Therefore, the core methodology of our theory is the application of a widely validated idea in physics—\textbf{Renormalization}. It is the bridge connecting our universe's discrete computational substrate with continuous mathematical methods.

\subsection{The Flow of Scales: Construction of the Effective Entropy Field}
We physicalize computational entropy as a discrete, fine-grained \textbf{fundamental entropy tensor field $\Sigma_0(P)$} at the \textbf{bit-scale}. This is the ontological cornerstone of our theory, the sole source of all higher-level physical realities. Any given physical process only interacts with an \textbf{``effective entropy field'' $\Sigma_{\text{eff}}(P; \lambda)$}, which is formed from $\Sigma_0(P)$ through a \textbf{``renormalization transformation'' (i.e., coarse-graining and redefinition of parameters)} at an \textbf{``effective scale'' $\lambda$} defined by the process's ``intrinsic scale'' (or observational energy). It should be noted that in our theory, renormalization is a statistically-based mathematical approximation method for studying discrete problems with continuous methods, and it is unrelated to the fundamental physical essence.

\subsection{The Effective Field of Focus in This Paper: $\Sigma(P)$ and its Multipole Structure}
This paper aims to explain gravity and quantum measurement, both of which are phenomena that occur at the \textbf{statistical scale}. Therefore, we focus on the effective entropy field at this scale, which, for the sake of simplicity, we denote as $\Sigma(P)$. We perform a \textbf{Multipole Expansion} on this emergent effective entropy field $\Sigma(P)$ at the statistical scale to analyze the different aspects (moments) of its geometric structure. These ``moments'' correspond to the different macroscopic physical phenomena we observe:

\begin{itemize}
    \item \textbf{Zeroth-Order Moment (Monopole Moment):} The \textbf{scalar trace $\sigma(P)$} of the effective entropy tensor field $\Sigma(P)$, which is the \textbf{effective entropy density}. Its physical effect is to determine the ``computational viscosity'' of spacetime, thereby giving rise to \textbf{gravitational time dilation}.
    
    \item \textbf{First-Order Moment (Dipole Moment):} The \textbf{first-order covariant derivative} of the effective entropy tensor field $\Sigma(P)$, whose simplest component is the \textbf{vector gradient $\nabla\sigma$}. Its physical effect is to generate an ``entropic pressure differential,'' thereby giving rise to \textbf{gravity} itself.
    
    \item \textbf{Second- and Higher-Order Moments (Quadrupole Moment, etc.):} The higher-order derivatives and intrinsic algebraic properties of the effective entropy tensor field $\Sigma(P)$, which describe its local \textbf{anisotropy}. We assert that this ``anisotropy'' of $\Sigma(P)$ is precisely the physical entity of the \textbf{``super-deterministic'' field} that serves as the background for quantum measurement.
\end{itemize}


% --- CHAPTER 3 ---
\section{The Entropic Dynamics of Quantum Events}

This chapter will provide a unified description of the two phases of a quantum event, attributing both to the interaction of particles with the ``structure of entropy.''

\subsection{The Ontology of a Particle: A Stable ``Low-Entropy Topological Structure''}
A fundamental particle (such as an electron) is an \textbf{``spacetime vortex''} (information soliton) with a fixed intrinsic structure and low entropy. All its quantum numbers (charge, spin, etc.) are manifestations of its unique \textbf{``low-entropy topological form''}.

\subsection{The ``Creation'' of Entanglement: Coupling with the ``Microscopic Structure'' of Entropy}
\begin{itemize}
    \item   \textbf{Mechanism:} In the event of particle pair creation, the two nascent ``low-entropy topological structures'' undergo a profound \textbf{coupling} with a \textbf{high-frequency, complex ``entropy field fluctuation''} that is microscopic in the local spacetime.
    \item   \textbf{Consequence:} This coupling process, like an ``entropic mold,'' \textbf{``forges''} the ``low-entropy topological forms'' of these two particles to be perfectly complementary, and ``locks'' their states to that \textbf{specific, yet unknowable to any local observer}, microscopic entropy fluctuation.
    \item   \textbf{Source of Randomness:} The unknowability of the \textbf{``microscopic structure'' of entropy} at the moment of creation endows the particle pair with a definite, yet unknown, initial state.
\end{itemize}

\subsection{The ``Measurement'' of a Quantum State: Alignment with the ``Macroscopic Structure'' of Entropy}
\begin{itemize}
    \item   \textbf{Mechanism:} A measurement apparatus forces the already ``forged'' particle to undergo a \textbf{``structural alignment''} between its own \textbf{``intrinsic low-entropy topological form''} and the \textbf{``macroscopic entropy structure'' (i.e., the background field $\Sigma(P)$)} of the apparatus's environment.
    \item   \textbf{Consequence:} The measurement outcome is the \textbf{deterministic} output of this ``structural alignment'' that corresponds to the lowest energy state (e.g., ``up'' or ``down'').
    \item   \textbf{Source of Determinism:} The measurement is based on the macroscopic entropy field environment where the instrument is located, therefore, its result is deterministic.
\end{itemize}

\subsection{The Essential Unity of Measurement and Creation: A Scale Emergence from ``1-to-2'' to ``1-to-N''}
We assert that ``measurement'' is not a new physical process fundamentally different from ``creation.'' Both are, in essence, the \textbf{same dynamics}—namely, \textbf{``Entanglement Spreading.''}

\begin{itemize}
    \item   \textbf{Entanglement Creation (``1-to-2''):} The preparation of an entangled state is the ``spreading'' and \textbf{preparation} of a source quantum state into a \textbf{microscopic entangled state of 2 particles}. This is a \textbf{microscopic process where coherence can be tracked}.
    \item   \textbf{Quantum Measurement (``1-to-N''):} A measurement, conversely, involves a microscopic system interacting with a macroscopic apparatus composed of \textbf{$N \approx 10^{23}$ particles}, causing its entangled state to \textbf{``avalanche-spread,''} thereby \textbf{giving rise to} a \textbf{classical macroscopic state composed of $10^{23}$ particles}.
\end{itemize}

\noindent The \textbf{``wave function collapse''} or the quantum-classical boundary is not a mysterious leap, but a fully physical, \textbf{statistics-based phase transition}. It marks the evolution of a system from a \textbf{``fluctuation-dominated'' microscopic quantum state}, via \textbf{``entanglement spreading,''} to an \textbf{``average-dominated'' macroscopic classical state}.


% --- CHAPTER 4 ---
\section{From Theory to Cosmology -- The Origin and Properties of the Background Entropy Field}

\subsection{The Origin of the Macroscopic Entropy Structure}
The macroscopic entropy structure field $\Sigma(P)$, which serves as the measurement background, emerges collectively from the contributions of all bit patterns within its \textbf{entire past light cone} through an \textbf{action integral}.

\subsection{The Dual Dynamics of the Entropy Field: A Game between Matter and Dark Energy}
The macroscopic structure and dynamic evolution of the $\Sigma(P)$ field originate from two opposing and perpetually competing physical processes:

\begin{itemize}
    \item   \textbf{The Effect of Matter:}
    \textbf{Low-entropy structures} defined by \textbf{matter} [information solitons]. By \textbf{consuming entropy}, matter attempts to establish and maintain ordered, stable ``structural islands'' within the computational field.

    \item   \textbf{The Effect of Dark Energy:}
    A \textbf{high-entropy substrate} defined by the \textbf{vacuum environment} itself. By \textbf{creating entropy}, dark energy manifests as the eternal, chaotic ``background boiling'' of the cosmic computational field.
\end{itemize}

\subsection{Gravity and the Coherent Decoding Field: "First-Order" and "Higher-Order" Emergence of the Same Low-Entropy Effect}
We assert that the ``low-entropy effect'' generated by the macroscopic body of \textbf{Earth} creates a special, ``purified'' region in the space surrounding it. Within this region, the intense, high-entropy ``noise'' from the external cosmic vacuum is significantly \textbf{shielded}.

This matter-dominated ``low-entropy shielded region'' gives rise to two distinct but homologous macroscopic physical effects, which can be understood as different moments of the $\Sigma(P)$ field:

\begin{itemize}
    \item   \textbf{The First-Order Effect is Gravity:}
    This is a manifestation of the \textbf{first-order structure [gradient]} of the $\Sigma(P)$ field. It is the most macroscopic and longest-range expression of the low-entropy effect of matter. Its action gives rise to gravity.

    \item   \textbf{The Higher-Order Effect is the Coherent Decoding Field:}
    This is a manifestation of the \textbf{higher-order structures [anisotropy, chirality, etc.]} of the $\Sigma(P)$ field. Within this shielded region, the fine structure of the entropy field becomes relatively stable and ordered, thus forming a \textbf{Coherent Decoding Field} capable of sustaining quantum entanglement correlations.
\end{itemize}

\noindent Because gravity is its \textbf{lower-order effect}, it is more prominent and has a wider range of influence on the macroscopic scale. The Coherent Decoding Field, being its \textbf{higher-order effect}, is more ``subtle'' and more sensitive to microscopic quantum processes [such as measurement].

\subsection{The Scale of Coherence: A Semi-Quantitative Inference from Experiment}
The \textbf{spacetime range} over which this ``Coherent Decoding Field''—sustained by Earth's low-entropy effect—can effectively shield the high-entropy effects of the external vacuum, i.e., the \textbf{coherent spacetime scale} [$\tau_c$ and $L_c$], is determined by the \textbf{strength ratio} between the \textbf{low-entropy effect of Earth} and the \textbf{high-entropy effect of the vacuum}.

We cannot calculate this range directly, but we can make a semi-quantitative inference about it using existing experimental data.

\begin{itemize}
    \item   \textbf{Principle of Spacetime Symmetry:} One of the cornerstones of our theory is spacetime symmetry. Therefore, the coherent spatial distance $L_c$ and the coherent time interval $\tau_c$ must be locked by the speed of light $c$:
    \textbf{$L_c = \tau_c \cdot c$}

    \item   \textbf{Constraint from the "Micius" Experiment:}
    China's "Micius" quantum science satellite has successfully distributed and maintained high-quality quantum entanglement between two ground stations separated by more than \textbf{1200 kilometers}.
    This irrefutable experimental fact sets a \textbf{solid experimental lower bound} for the coherence scale of our theory.

    \item   \textbf{Our Prediction:}
    The fact that entanglement correlation can still be maintained over a spatial distance of $L_{\text{exp}} \approx 1200$ km implies that $L_{\text{exp}}$ must be \textbf{within the range of the coherence length $L_c$}.
    From this, we can make a \textbf{semi-quantitative prediction} for the coherence time $\tau_c$:
    \[ \tau_c = L_c / c \geq L_{\text{exp}} / c \approx \frac{1200 \text{ km}}{300,000 \text{ km/s}} = 0.004 \text{ seconds} \]
    
    \begin{quote}
        \textbf{We predict that the coherence time $\tau_c$ of our local universe should be at least 4 milliseconds.} This falsifiable lower bound provides a clear \textbf{theoretical benchmark} that must be surpassed by all future experiments attempting to test fundamental physics through ``time-delayed correlation measurements.''
    \end{quote}
\end{itemize}


% --- CHAPTER 5 ---
\section{A Decisive Physical Experiment}

\subsection{Objective}
This experiment aims to go beyond traditional Bell's inequality tests. Our goal is no longer to verify the existence of quantum correlations themselves, but to \textbf{directly measure} the coherence time $\tau_c$ of the macroscopic \textbf{``cosmic entropy structure field $\Sigma(P)$''} which, in our theory, is the root of quantum phenomena—that is, how long the ``memory'' of this field can be maintained.

\subsection{Experimental Method: Time-Delayed Entanglement Correlation Measurement with a ``Zero-Delay'' Control Group}
The core of this experimental design is to \textbf{alternatingly} conduct measurements in two different modes within the same experimental setup, in order to precisely separate the true physical effect from potential systematic errors.

\begin{itemize}
    \item   \textbf{Experimental Group:} To measure the physical effect of the temporal evolution of the cosmic background field on entanglement strength.
    \begin{itemize}
        \item   Alice performs a measurement on particle A at time $t_A$.
        \item   Bob performs a measurement on particle B after a controllable time delay $\Delta t$, at $t_B = t_A + \Delta t$.
    \end{itemize}

    \item   \textbf{Control Group:} To calibrate and quantify the systematic errors of the experimental apparatus itself that arise over time.
    \begin{itemize}
        \item   Alice and Bob \textbf{simultaneously} perform a measurement on an entangled pair at a time $t + \Delta t$ [which has evolved for the same duration], for the purpose of error calibration.
    \end{itemize}
\end{itemize}

\subsection{Experimental Conclusion and the Ultimate Verdict on Our Theory}
The final, control-group-calibrated \textbf{purely physical correlation decay curve, $C_{\text{phys}}(\Delta t)$}, will provide a \textbf{single, decisive verdict} on our theory.

Based on the first principles of our theory [such as the equivalence principle and entropy dynamics], we make only \textbf{one} core, semi-quantitative prediction: the coherence time $\tau_c$ should lie within a \textbf{specific macroscopic time window} dominated by the physical scale of the Earth. Based on the experimental data from the ``Micius'' quantum science satellite, we have already set an \textbf{experimental lower bound} for this window, namely $\tau_c \geq 4\text{ms}$.

Therefore, there are only two logically mutually exclusive possible outcomes for the experiment:

\begin{itemize}
    \item   \textbf{Possible Outcome 1: Theory Corroborated}
    \begin{itemize}
        \item   \textbf{Observation:} The experiment observes that the characteristic decay time $\tau_c$ of the correlation $C_{\text{phys}}(\Delta t)$ falls significantly within the macroscopic time window of \textbf{milliseconds to seconds}.
        \item   \textbf{Theoretical Verdict:} This would \textbf{strongly corroborate} the entire framework of our theory. It would mean that we have not only correctly identified the physical carrier of superdeterminism [the $\Sigma(P)$ field] but have also successfully predicted its rate of evolution.
    \end{itemize}

    \item   \textbf{Possible Outcome 2: Theory Falsified}
    \begin{itemize}
        \item   \textbf{Observation:} The experiment observes that $\tau_c$ falls significantly outside our predicted window. For example, $\tau_c$ is much shorter than the millisecond level [complete decoherence on the nanosecond or microsecond scale], or it is so long that no observable decay occurs within the accessible time frame of the experiment [e.g., several hours or more].
        \item   \textbf{Theoretical Verdict:} This would \textbf{clearly falsify} the core mechanism of our theory. It would imply the existence of other physical mechanisms governing quantum correlations that our theory has not foreseen, or that the properties of the background field are completely different from our predictions.
    \end{itemize}
\end{itemize}



% --- CHAPTER 6 ---
\section{Conclusion — Physics as ``The Structural Geometry of Entropy''}

We have proposed a unified, deterministic model of quantum entanglement based entirely on ``structured entropy.'' In this model, the mysteries of the quantum world—including its probabilistic nature, non-local correlations, and ``measurement collapse''—are reduced to a more profound and physical unified process: \textbf{the dynamic interaction between matter (as a stable, low-entropy topological structure) and the universal entropy field at different structural scales.}

\begin{itemize}
    \item   The \textbf{creation} of entanglement is the process by which particles are ``forged'' in the \textbf{microscopic structure (small-scale fluctuations)} of entropy, which injects ``unknowable order'' into them.
    \item   The \textbf{measurement} of entanglement is the process by which this particle's state is ``read out'' against the \textbf{macroscopic structure (large-scale fluctuations)} of entropy, a statistics-based phase transition from the ``microscopically unknowable'' to the ``macroscopically definite.''
\end{itemize}

\noindent This perspective ultimately transforms the core questions of quantum mechanics into the study of a \textbf{single, yet infinitely complex, physical entity—the universal entropy field}. We no longer merely ask ``What exists?'' but rather, ``What structure does that which exists (entropy) possess?''

\noindent Thus, our theory points to a completely new picture: \textbf{Gravity and quantum entanglement are, in essence, different manifestations of ``The Structural Geometry of Entropy.''}

\noindent The significance of our proposed ``time-delayed entanglement experiment'' therefore becomes immensely profound. It is no longer just a test of a specific model of entanglement; it is \textbf{using the most precise ``probe'' of quantum phenomena to directly measure the dynamic properties of the universe's most grand ``entropic structure.''} This experiment will deliver the final, decisive verdict on this path of exploration that unifies quantum information, gravitational physics, and cosmology under the central problem of ``the structural scales of entropy.''

\end{document}
