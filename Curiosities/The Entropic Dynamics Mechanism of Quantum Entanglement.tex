\documentclass[11pt]{article}

% --- PACKAGES ---
\usepackage[utf8]{inputenc}
\usepackage{amsmath}
\usepackage{amssymb}
\usepackage[a4paper, margin=1in]{geometry}

% --- TITLE AND AUTHOR ---
\title{\textbf{The Entropic Dynamics Mechanism of Quantum Entanglement}}
\author{Haifeng Qiu} 


\begin{document}

\maketitle
\thispagestyle{empty} % No page number on the title page
\newpage
\setcounter{page}{1}


% --- CHAPTER 1 ---
\section{Introduction — The Scale Emergence of Physics}

\subsection{A Unified Starting Point: From Bits to the Structural Scales of the Universe}

The framework of ``Computational Realism'' that we construct has at its core the following assertion: The entirety of physical reality as we know it consists of effective dynamics emerging at multiple, distinct structural scales from a single, deterministically ruled \textbf{bit computation field}. We divide this into four scales:

\begin{enumerate}
    \item   \textbf{The Bit-Scale:} The most fundamental, discrete computational substrate of the universe. Its dynamics are governed by a unique \texttt{Rule}, manifesting as pseudo-random chaos, which is the ultimate origin of all physical phenomena.
    
    \item   \textbf{The Particle-Scale:} Stable, local patterns with specific \textbf{topological structures}, formed by the self-organization of bits. Its dynamics (topological dynamics) are responsible for explaining the generation mechanism of \textbf{fundamental particles} and their \textbf{strong, weak, and electromagnetic interactions}.
    
    \item   \textbf{The Microscopic Scale of Entropic Effects:} This is the manifestation of the chaos at the \textbf{Bit-Scale} as \textbf{high-frequency fluctuations} in microscopic regions of spacetime. Its dynamics are responsible for injecting definite, yet unknowable, initial conditions into the \textbf{creation of quantum events}.
    
    \item   \textbf{The Statistical Scale of Entropic Effects:} This is the \textbf{smooth, classical background entropy structure field} that emerges after the chaos of the \textbf{Bit-Scale} is \textbf{statistically averaged} over macroscopic spacetime regions. Its dynamics (entropic dynamics) lead to the effects of \textbf{gravity} and \textbf{quantum measurement}.
\end{enumerate}

\subsection{Positioning and Core Thesis of This Paper}

The complete theory of Computational Realism aims to provide a unified explanation for all the emergence mechanisms, from the Bit-Scale to all higher scales. \textbf{This paper, however, will focus on the third and fourth scales mentioned above—namely, the two scales of ``entropic effects''}—and, based on this, provide a unified mechanistic explanation for \textbf{gravity} and \textbf{quantum entanglement}.

For the clarity of the arguments in this paper, the terms \textbf{``microscopic''} and \textbf{``macroscopic''} used hereafter will be \textbf{specifically and locally} defined as follows:
\begin{itemize}
    \item   \textbf{``Microscopic'' in this paper} refers specifically to the \textbf{``Microscopic Scale of Entropic Effects,''} i.e., the entropy field fluctuations that couple with quantum creation.
    \item   \textbf{``Macroscopic'' in this paper} refers specifically to the \textbf{``Statistical Scale of Entropic Effects,''} i.e., the classical entropy structure field that serves as the background for quantum measurement.
\end{itemize}

\noindent Based on this qualification, the core thesis of this paper is: The mystery of quantum entanglement can be fully understood as the inevitable deterministic consequence of the interaction between a \textbf{particle (a low-entropy structure emerging at the ``Particle-Scale'')} and the universal entropy field at its \textbf{``Microscopic Scale of Entropic Effects'' (influencing its creation)} and its \textbf{``Statistical Scale of Entropic Effects'' (influencing its measurement)}.


% --- CHAPTER 2 ---
\section{The Ontology of Entropy — A Multi-Scale Effective Field}

\subsection{Fundamental Postulate: Computational Entropy as the Sole Reality}
The cornerstone of our theory is that the ultimate measure of reality is \textbf{Computational Entropy (Kolmogorov Complexity, $S_C$)}. It measures the incompressibility of a pattern.

\subsection{Application of the Renormalization Idea: From the Fundamental Field to the Effective Field}
In order to study a discrete, fine-grained field defined at the bit-scale using continuous mathematics, we must define effective scales. Therefore, the core methodology of our theory is the application of a widely validated idea in physics—\textbf{Renormalization}. It is the bridge connecting our universe's discrete computational substrate with continuous mathematical methods.

\subsection{The Flow of Scales: Construction of the Effective Entropy Field}
We physicalize computational entropy as a discrete, fine-grained \textbf{fundamental entropy tensor field $\Sigma_0(P)$} at the \textbf{bit-scale}. This is the ontological cornerstone of our theory, the sole source of all higher-level physical realities. Any given physical process only interacts with an \textbf{``effective entropy field'' $\Sigma_{\text{eff}}(P; \lambda)$}, which is formed from $\Sigma_0(P)$ through a \textbf{``renormalization transformation'' (i.e., coarse-graining and redefinition of parameters)} at an \textbf{``effective scale'' $\lambda$} defined by the process's ``intrinsic scale'' (or observational energy).

\subsection{The Effective Field of Focus in This Paper: $\Sigma(P)$ and its Multipole Structure}
This paper aims to explain gravity and quantum measurement, both of which are phenomena that occur at the \textbf{statistical scale}. Therefore, we focus on the effective entropy field at this scale, which, for the sake of simplicity, we denote as $\Sigma(P)$. We perform a \textbf{Multipole Expansion} on this emergent effective entropy field $\Sigma(P)$ at the statistical scale to analyze the different aspects (moments) of its geometric structure. These ``moments'' correspond to the different macroscopic physical phenomena we observe:

\begin{itemize}
    \item \textbf{Zeroth-Order Moment (Monopole Moment):} The \textbf{scalar trace $\sigma(P)$} of the effective entropy tensor field $\Sigma(P)$, which is the \textbf{effective entropy density}. Its physical effect is to determine the ``computational viscosity'' of spacetime, thereby giving rise to \textbf{gravitational time dilation}.
    
    \item \textbf{First-Order Moment (Dipole Moment):} The \textbf{first-order covariant derivative} of the effective entropy tensor field $\Sigma(P)$, whose simplest component is the \textbf{vector gradient $\nabla\sigma$}. Its physical effect is to generate an ``entropic pressure differential,'' thereby giving rise to \textbf{gravity} itself.
    
    \item \textbf{Second- and Higher-Order Moments (Quadrupole Moment, etc.):} The higher-order derivatives and intrinsic algebraic properties of the effective entropy tensor field $\Sigma(P)$, which describe its local \textbf{anisotropy}. We assert that this ``anisotropy'' of $\Sigma(P)$ is precisely the physical entity of the \textbf{``super-deterministic'' field} that serves as the background for quantum measurement.
\end{itemize}


% --- CHAPTER 3 ---
\section{The Entropic Dynamics of Quantum Events}

This chapter will provide a unified description of the two phases of a quantum event, attributing both to the interaction of particles with the ``structure of entropy.''

\subsection{The Ontology of a Particle: A Stable ``Low-Entropy Topological Structure''}
A fundamental particle (such as an electron) is an \textbf{``spacetime vortex''} (information soliton) with a fixed intrinsic structure and low entropy. All its quantum numbers (charge, spin, etc.) are manifestations of its unique \textbf{``low-entropy topological form''}.

\subsection{The ``Creation'' of Entanglement: Coupling with the ``Microscopic Structure'' of Entropy}
\begin{itemize}
    \item   \textbf{Mechanism:} In the event of particle pair creation, the two nascent ``low-entropy topological structures'' undergo a profound \textbf{coupling} with a \textbf{high-frequency, complex ``entropy field fluctuation''} that is microscopic in the local spacetime.
    \item   \textbf{Consequence:} This coupling process, like an ``entropic mold,'' \textbf{``forges''} the ``low-entropy topological forms'' of these two particles to be perfectly complementary, and ``locks'' their states to that \textbf{specific, yet unknowable to any local observer}, microscopic entropy fluctuation.
    \item   \textbf{Source of Randomness:} The unknowability of the \textbf{``microscopic structure'' of entropy} at the moment of creation endows the particle pair with a definite, yet unknown, initial state.
\end{itemize}

\subsection{The ``Measurement'' of a Quantum State: Alignment with the ``Macroscopic Structure'' of Entropy}
\begin{itemize}
    \item   \textbf{Mechanism:} A measurement apparatus forces the already ``forged'' particle to undergo a \textbf{``structural alignment''} between its own \textbf{``intrinsic low-entropy topological form''} and the \textbf{``macroscopic entropy structure'' (i.e., the background field $\Sigma(P)$)} of the apparatus's environment.
    \item   \textbf{Consequence:} The measurement outcome is the \textbf{deterministic} output of this ``structural alignment'' that corresponds to the lowest energy state (e.g., ``up'' or ``down'').
    \item   \textbf{Source of Determinism:} The measurement is based on the macroscopic, slowly varying entropy field environment, rendering its outcome deterministic.
\end{itemize}

\subsection{The Essential Unity of Measurement and Creation: A Scale Emergence from ``1-to-2'' to ``1-to-N''}
We assert that ``measurement'' is not a new physical process fundamentally different from ``creation.'' Both are, in essence, the \textbf{same dynamics}—namely, \textbf{``Entanglement Spreading.''}

\begin{itemize}
    \item   \textbf{Entanglement Creation (``1-to-2''):} The preparation of an entangled state is the ``spreading'' and \textbf{preparation} of a source quantum state into a \textbf{microscopic entangled state of 2 particles}. This is a \textbf{microscopic process where coherence can be tracked}.
    \item   \textbf{Quantum Measurement (``1-to-N''):} A measurement, conversely, involves a microscopic system interacting with a macroscopic apparatus composed of \textbf{$N \approx 10^{23}$ particles}, causing its entangled state to \textbf{``avalanche-spread,''} thereby \textbf{giving rise to} a \textbf{classical macroscopic state composed of $10^{23}$ particles}.
\end{itemize}

\noindent The \textbf{``wave function collapse''} or the quantum-classical boundary is not a mysterious leap, but a fully physical, \textbf{statistics-based phase transition}. It marks the evolution of a system from a \textbf{``fluctuation-dominated'' microscopic quantum state}, via \textbf{``entanglement spreading,''} to an \textbf{``average-dominated'' macroscopic classical state}.



% --- CHAPTER 4 ---
\section{From Theory to Cosmology — The Origin and Properties of the Background Entropy Field}

\subsection{Origin of the Macroscopic Entropy Structure}
The macroscopic entropy structure field $\Sigma(P)$, which serves as the background for measurement, emerges collectively from the contributions of all matter sources within its \textbf{entire past light cone}, via an \textbf{action integral}. The state of $\Sigma(P)$ encodes the complete history of the universe.

\subsection{The Dual Dynamics of the Entropy Field}
The dynamic evolution of the $\Sigma(P)$ field originates from two distinct types of physical processes:
\begin{itemize}
    \item   \textbf{Ordered ``Mechanical'' Effects:} \textbf{Periodic variations} caused by the collective motions of celestial bodies (rotation, revolution).
    \item   \textbf{Disordered ``Thermodynamic'' Effects:} \textbf{Random variations} caused by the asymmetric distribution and motion of mass within celestial bodies (entropy increase).
\end{itemize}

\subsection{Slow Spatial Variation and Signal-to-Noise Ratio}
The dominant contributions that determine the value of $\Sigma(P)$ come from macroscopic celestial bodies (like the Earth and the Sun). This ensures that the $\Sigma(P)$ field is \textbf{highly spatially coherent} at the laboratory scale. The strength of the ordered effects (the signal), dominated by distant celestial bodies, far exceeds the random effects (the noise) caused by near-field matter. This \textbf{extremely high ``signal-to-noise ratio''} is the foundation upon which the correlations of entanglement can be maintained.



% --- CHAPTER 5 ---
\section{A Decisive Physical Experiment}

\subsection{Experimental Objective}
This experiment aims to go beyond Bell's theorem. Instead of verifying the correlations themselves, it seeks to \textbf{directly measure the coherence time $\tau_c$ of the universe's ``macroscopic entropy structure field $\Sigma(P)$''}, i.e., how long the ``memory'' of this field can be maintained.


\subsection{The a priori Prediction of the Theory: The Coherence Time of the Background Field $\Sigma(P)$}
We must first clarify the physical sources of the dynamic changes in the $\Sigma(P)$ field. Since our laboratory is \textbf{co-moving} with the Earth itself, the Earth's rotation per se does not produce a directly measurable temporal variation within the laboratory. Therefore, the measurable variations of the $\Sigma(P)$ field must originate from two aspects: \textbf{external celestial bodies} (primarily the Sun and the Moon) and the \textbf{Earth's internal dynamics}.
\begin{enumerate}
    \item   \textbf{Periodic ``Tidal'' Effects:} Caused by the periodic motions (rotation and revolution) of the Earth relative to the Sun and the Moon. This is a \textbf{calculable and predominantly periodic ``external'' modulation}.
    \item   \textbf{Random ``Geocentric'' Effects:} Caused by the \textbf{internal, irreversible ``thermodynamic entropy increase''} processes of the Earth (such as mantle convection), which lead to \textbf{random variations in the asymmetric ``thermodynamic multipole moment''}. This constitutes an \textbf{unpredictable ``internal'' noise}.
\end{enumerate}

\noindent The coherence time of entanglement, $\tau_c$—whose reciprocal is the decoherence rate—will be dominated by whichever of these two effects is \textbf{stronger, faster-changing, and more random}.
Based on physical estimations, the period of variation for the tidal effects caused by the Sun and the Moon is \textbf{extremely macroscopic} (hours/days/months). Meanwhile, the characteristic timescale for the macroscopic multipole moment variations caused by the Earth's internal thermodynamic processes is also considered to be \textbf{extremely slow} (on geological timescales).

\vspace{1em} % Adds a bit of vertical space before the main prediction
\noindent From this, our theory makes a final core prediction, based on an analysis of the physical mechanisms:

\begin{center}
    \textbf{Whether it is the external ``tidal'' effects or the internal ``geocentric'' effects, the characteristic timescales driving the changes in the $\Sigma(P)$ field are all extremely macroscopic. Therefore, we predict that in the ``time-delayed entanglement experiment'' described below, the $\Sigma(P)$ field will appear to be extremely stable, and its coherence time, $\tau_c$, will be a very long, macroscopic timescale.}
\end{center}


\subsection{Experimental Method: Time-Delayed Entanglement Correlation Measurement with a ``Zero-Delay'' Control Group}
The core of this experimental design is to alternate between two different measurement modes within the same experimental setup, in order to \textbf{precisely separate the true physical effect from experimental systematic errors}.
\begin{itemize}
    \item   \textbf{Experimental Group:} Alice measures at time $t_A$, and Bob measures at time $t_B = t_A + \Delta t$, to measure the entanglement strength as it evolves with the cosmic background field over time.
    \item   \textbf{Control Group:} Alice and Bob conduct a measurement \textbf{simultaneously} at time $t + \Delta t$, to calibrate the systematic error of the experimental apparatus itself that arises with the delay $\Delta t$.
\end{itemize}


\subsection{The Experimental Conclusion and the Ultimate Adjudication of Our Theory}

The resulting pure physical correlation decay curve $C(\Delta t)$ will provide a \textbf{unique and decisive adjudication} for our theory. Based on our analysis, derived from first principles (the equivalence principle, entropic dynamics), that ``the $\Sigma(P)$ field is absolutely dominated by Earth's macroscopic dynamics,'' our theory makes only \textbf{one} core prediction. Therefore, there are only two possible outcomes for the experiment:

\begin{itemize}
    \item   \textbf{Possible Outcome One: Confirmation of the Theory ($C(\Delta t)$ decays slowly)}
        \begin{itemize}
            \item   \textbf{Observation:} The experiment observes that the correlation $C(\Delta t)$ shows \textbf{almost no measurable decay} over timescales of hours, days, or even longer ($\tau_c$ is extremely long). Within the measurable range of $\Delta t$, the curve is \textbf{approximately a flat line}.
            \item   \textbf{Theoretical Adjudication:} This would constitute an \textbf{extremely powerful and decisive confirmation} of our entire theoretical framework—from the ontology of the entropy field, to the super-deterministic mechanism, and to the final inference that the $\Sigma(P)$ field is dominated by Earth's dynamics. It would prove that the background for quantum phenomena in our universe is indeed a stable, classical, and predictable macroscopic field, much like the geomagnetic field.
        \end{itemize}

    \item   \textbf{Possible Outcome Two: Falsification of the Theory ($C(\Delta t)$ decays rapidly)}
        \begin{itemize}
            \item   \textbf{Observation:} The experiment observes that the correlation $C(\Delta t)$ undergoes a \textbf{significant and reproducible decay} on \textbf{any macroscopic timescale shorter than ``hours'' (e.g., seconds or minutes)}.
            \item   \textbf{Theoretical Adjudication:} This would \textbf{irrefutably and decisively falsify} the core prediction of our theory and would likely mean that at least one of the following is true:
                \begin{enumerate}
                    \item   Our inference that ``the $\Sigma(P)$ field is dominated by Earth's macroscopic dynamics'' is \textbf{incorrect}. There exists some unforeseen, more powerful, and faster-varying source of ``noise.''
                    \item   Our entire theoretical model of the ``macroscopic entropy structure field'' is fundamentally \textbf{incorrect} or \textbf{incomplete}.
                    \item   The nature of quantum entanglement follows a physical mechanism entirely different from the one we have proposed.
                \end{enumerate}
        \end{itemize}
\end{itemize}


% --- CHAPTER 6 ---
\section{Conclusion — Physics as ``The Structural Geometry of Entropy''}

We have proposed a unified, deterministic model of quantum entanglement based entirely on ``structured entropy.'' In this model, the mysteries of the quantum world—including its probabilistic nature, non-local correlations, and ``measurement collapse''—are reduced to a more profound and physical unified process: \textbf{the dynamic interaction between matter (as a stable, low-entropy topological structure) and the universal entropy field at different structural scales.}

\begin{itemize}
    \item   The \textbf{creation} of entanglement is the process by which particles are ``forged'' in the \textbf{microscopic structure (small-scale fluctuations)} of entropy, which injects ``unknowable order'' into them.
    \item   The \textbf{measurement} of entanglement is the process by which this particle's state is ``read out'' against the \textbf{macroscopic structure (large-scale fluctuations)} of entropy, a statistics-based phase transition from the ``microscopically unknowable'' to the ``macroscopically definite.''
\end{itemize}

\noindent This perspective ultimately transforms the core questions of quantum mechanics into the study of a \textbf{single, yet infinitely complex, physical entity—the universal entropy field}. We no longer merely ask ``What exists?'' but rather, ``What structure does that which exists (entropy) possess?''

\noindent Thus, our theory points to a completely new picture: \textbf{Gravity and quantum entanglement are, in essence, different manifestations of ``The Structural Geometry of Entropy.''}

\noindent The significance of our proposed ``time-delayed entanglement experiment'' therefore becomes immensely profound. It is no longer just a test of a specific model of entanglement; it is \textbf{using the most precise ``probe'' of quantum phenomena to directly measure the dynamic properties of the universe's most grand ``entropic structure.''} This experiment will deliver the final, decisive verdict on this path of exploration that unifies quantum information, gravitational physics, and cosmology under the central problem of ``the structural scales of entropy.''

\end{document}
