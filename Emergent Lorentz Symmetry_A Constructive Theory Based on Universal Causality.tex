\documentclass[11pt, a4paper]{article}

% --- Preamble ---
\usepackage[top=2.5cm, bottom=2.5cm, left=2.5cm, right=2.5cm]{geometry}
\usepackage{amsmath, amssymb}
\usepackage{hyperref}
\hypersetup{colorlinks=true, linkcolor=blue, citecolor=blue, urlcolor=blue}
\usepackage{xurl}

% --- Document Information ---
\title{\textbf{Emergent Lorentz Symmetry: A Constructive Theory Based on Universal Causality}}
\author{Haifeng Qiu}


% --- Document Body ---
\begin{document}

\maketitle

% --- Abstract ---
\begin{abstract}
\noindent This paper aims to provide a new, constructive interpretation for Special Relativity (SR). We challenge the traditional view that Lorentz symmetry is an inherent geometric property of spacetime and propose that it may be understood as a dynamical consequence of a deeper ontological principle. This principle is: in a logically self-consistent universe, all entities must adhere to \textit{the same set of universal causal propagation constraints (an upper speed limit c)}. We argue that the Lorentz transformation is the dynamical behavior that must necessarily emerge for any extended pattern that relies on internal causal coordination to maintain its ``identity'' while conforming to this universal constraint. Through a geometric derivation based on symmetry principles and ``internal causal self-consistency,'' we demonstrate how time dilation and length contraction become logical necessities for maintaining such consistency, wherein the invariance of transverse dimensions is shown to be a direct consequence of the isotropy of the causal framework. Based on this, we reconstruct the principle of the ``constancy of the speed of light'' as an operational ``measurement invariance.'' The core thesis of this paper is that an ontological framework based on ``dynamical emergence'' not only provides a deeper foundation for SR but also may offer a new path for reconceptualizing frontier philosophical problems such as black hole singularities and the information paradox.
\end{abstract}

% --- Sections ---
\section{Introduction: Principle, Construction, and the Foundational Crisis in Physics}

Albert Einstein's Special Relativity (SR) is a paragon of a ``principle theory'' \cite{Einstein1905, Brown2005}. It logically derives its entire theoretical system from two concise and powerful universal principles. However, principle theories, while successful, leave in suspense the physical mechanisms behind their principles. This explanatory ``incompleteness'' is acceptable within the framework of SR, but when its ideas are extended to General Relativity (GR) and applied to extreme scenarios like black holes, it gives rise to profound philosophical and physical crises—such as the appearance of singularities and the information paradox.

This paper suggests that the root of these crises may be traced back to the choice of a ``principle'' rather than a ``constructive'' explanatory path for Lorentz symmetry. A constructive path seeks to ``build'' or ``emerge'' macroscopic physical laws from more fundamental, postulated physical entities and dynamical processes. This line of thought has a long history, with precursors including Poincaré \cite{Poincare1905}, receiving a powerful philosophical defense in the 20th century from physicists like John Bell \cite{Bell1976}, and being developed in contemporary times into a detailed ``common cause explanation'' by Janssen and others \cite{Janssen2002}.

This paper aims to propose a more universal philosophical framework along this constructive path. Our entire argument is rooted in a core ontological principle: \textit{the universe, as a logically self-consistent existence, must have its internal causal relations adhere to a single, unified logic.} We assert that there exists a universal upper limit to the speed of causal propagation, c, which is not merely a special rule for ``light'' but a fundamental constraint for the universe as a whole to maintain its own logical consistency.

The reason we undertake such a thorough reconstruction of the foundations of SR is not only for philosophical self-consistency but also because it offers a completely new conceptual path for resolving the profound philosophical dilemmas faced by GR. We believe that a correct ontology of spacetime must be able to maintain its explanatory power across all physical scales, and this idea of ``emergence'' is being actively explored at the frontiers of contemporary physics (e.g., in ``analogue gravity'' research) \cite{Barcelo2005}. This constructive interpretation of SR is the first step toward establishing such a more powerful framework.

\section{Ontological Postulates of the Model: A Unified Causal Framework}

Our entire derivation is built upon the following two fundamental postulates.

\subsection{Postulate I: The Universal Causal Framework}
We postulate the existence of a \textit{Universal Causal Framework} that serves as the background for all physical phenomena. This is not a substantial ``medium'' (like the ether), but a \textit{governing, non-substantial framework} or a \textit{meta-law} representing the intrinsic causal structure of the universe as a whole. Its fundamental property is that the propagation of any causal influence has an isotropic, absolute upper speed limit c within this framework.

\textbf{Defending the metaphysical status of this postulate is crucial to the paper's philosophical thesis.} We explicitly reject the Humean view that laws of nature are merely the best systematic summaries of physical events. For a constructive theory, laws must have explanatory priority; they must ``govern'' or ``constrain'' the behavior of matter, not merely ``describe'' it. We also avoid traditional Platonism, which posits laws as abstract objects in a separate realm, thereby facing the problem of how abstract entities interact with the physical world.

Instead, our position is closer to that advocated by Tim Maudlin, who treats laws of nature as governing, fundamental entities \cite{Maudlin2007}. However, we qualify this concept by emphasizing its ``non-substantiality.'' This ``framework'' has no physical properties like energy or momentum; it is more akin to a set of ``rules of the game'' or a ``causal grammar'' that dictates the logical constraints any possible dynamical process in the universe must follow. Its ``governing power'' derives not from a physical force, but from its fundamental status as the logic of the universe's existence. We adopt this metaphysical stance because only if this causal framework is ``governing'' rather than ``descriptive'' can it provide a solid ontological foundation for why matter is ``compelled'' to exhibit emergent Lorentz covariance. The \textbf{isotropy} of this framework is its key feature and will play a central role in our derivation.

\subsection{Postulate II: Matter as Causally Self-Consistent Patterns}
We postulate that stable, extended material entities are, ontologically, dynamic, self-sustaining stable patterns. Their ``identity'' fundamentally depends on continuous causal coordination among their internal parts. According to Postulate I, this internal causal coordination must also proceed at the speed c.

The ``causal coordination'' spoken of here is, at the most fundamental level, \textit{pre-metric}. From this, we can further deduce a key property of stable patterns: the \textit{Principle of Unified Transformation}. A stable complex pattern (like a proton) can be understood as a hierarchical system composed of a multitude of different, more basic ``causal loops'' in stable harmonic relationships. The ``unity'' here does not refer to the identity of frequency, but to the \textbf{unity of transformation}. To maintain the ``identity'' of the entire pattern—that is, to maintain the constant harmonic structure and phase relations among its constituent causal loops—all these basic loops, regardless of their intrinsic periods, must undergo dynamical adjustments governed by the same, universal factor. If the periods of its internal causal loops in different dimensions (which can be seen as the pattern's ``intrinsic clocks'', an idea that resonates with de Broglie \cite{deBroglie1930}) were to change by different proportions, the harmonic relationship would be altered, and the pattern's structure at time t would no longer be a precise replica of its structure at $t - \Delta t$. By definition, it would no longer be the same ``stable'' pattern. Therefore, for any self-consistent pattern to exist as a stable object, the adjustment of its internal dynamical rhythms in all dimensions \textit{must} be proportional. This principle is not an additional assumption but a direct \textit{analytic entailment} of our concept of a ``stable, self-consistent pattern.''

\section{Derivation I: The Argument from Internal Causal Self-Consistency}

In this section, we will conduct a purely logical and geometric deduction from the perspective of an information pattern itself, arguing that it must adjust its form and rhythm when in motion to maintain its internal causal self-consistency.

\subsection{Transverse Self-Consistency: The Emergence of Time Dilation}
A stable information pattern must be self-consistent in all dimensions. We first analyze its structure perpendicular to the direction of motion (transverse).

\begin{enumerate}
    \item \textbf{Intrinsic Transverse Properties:} When the pattern is at rest relative to the causal framework, it has a structural dimension $H_0$ in the y-direction. A causal signal making a round trip up and down within it defines its \textbf{intrinsic transverse period $T_{0,H} = 2H_0 / c$}.
    \item \textbf{Dynamical Consequence of Motion:} When the pattern moves with velocity $v$ in the x-direction, to maintain the causal self-consistency of this up-and-down loop, the internal signal must travel along a diagonal path in the framework. Let the dynamical period of the pattern's transverse loop be $T_H$ and its transverse dimension be $H$ while in motion. According to the Euclidean geometry of the framework (Pythagorean theorem), we necessarily obtain the relation:
    \begin{equation}
        (c \cdot T_H / 2)^2 = (v \cdot T_H / 2)^2 + H^2
    \end{equation}
    \item \textbf{Inference from the Principle of Symmetry:} Here, we must appeal to a principle more fundamental than ``internal unity,'' namely a symmetry argument rooted in \textbf{Postulate I (the isotropy of the Universal Causal Framework)}. Since the pattern's motion occurs entirely in the x-direction, there exists no physical reason or mechanism that could provide a sufficient reason (Principle of Sufficient Reason) for a change in size in the y-direction (or z-direction), which is perpendicular to the motion. According to Curie's principle, a cause in the x-direction cannot produce an unmotivated physical effect in an orthogonal direction (y-direction). Therefore, we must conclude that the pattern's transverse dimensions must remain unchanged during the dynamical adjustment, i.e., $H = H_0$. Any other possibility (whether contraction or expansion) would constitute an unjustified violation of the causal framework's isotropy.
    \item \textbf{The Emergence of Time Dilation:} Substituting $H = H_0$ into equation (1) and solving for $T_H$:
    \begin{gather}
        c^2 \frac{T_H^2}{4} = v^2 \frac{T_H^2}{4} + H_0^2 \\
        T_H = \frac{2 H_0}{c\sqrt{1 - v^2/c^2}}
    \end{gather}
    We know the period at rest is $T_{0,H} = 2H_0/c$, therefore:
    \begin{equation}
        T_H = \frac{T_{0,H}}{\sqrt{1 - v^2/c^2}} = \gamma \cdot T_{0,H}
    \end{equation}
    where $\gamma = 1 / \sqrt{1 - v^2/c^2}$ is the Lorentz factor. The direct dynamical consequence of this geometric constraint is that the pattern's \textbf{internal rhythm} must slow down by a factor of $\gamma$. This constitutes the dynamical origin of what we observe as \textbf{time dilation}.
\end{enumerate}

\subsection{Longitudinal Self-Consistency: The Emergence of Length Contraction}
Now, we analyze the pattern's structure in the direction of motion (longitudinal). According to the ``Principle of Unified Transformation'' we established in Section 2.2, we must have: the transformation factor for the longitudinal period must be identical to that for the transverse period, which is $\gamma$. Thus, the relationship between the longitudinal period in motion, $T_L$, and the longitudinal period at rest, $T_{0,L}$, is $T_L = \gamma \cdot T_{0,L}$.

Let the longitudinal dimension of the pattern in motion be $L$. For an internal causal signal to complete a round trip, there is a time difference in ``catching up'' with and ``meeting'' the front and back ends of the pattern relative to the framework. The time for the signal to travel forward is $t_1 = L / (c-v)$, and the time to travel backward is $t_2 = L / (c+v)$. Therefore, its longitudinal loop period is:
\begin{equation}
    T_L = t_1 + t_2 = \frac{L}{c-v} + \frac{L}{c+v} = \frac{2Lc}{c^2-v^2} = \frac{2L}{c}\gamma^2
\end{equation}
Now we equate the two expressions for $T_L$:
\begin{equation}
    \frac{2L}{c}\gamma^2 = \gamma \cdot T_{0,L}
\end{equation}
Substituting $T_{0,L}=2L_0/c$ (where $L_0$ is the rest length) and solving for $L$:
\begin{gather}
    \frac{2L}{c}\gamma^2 = \gamma \cdot \frac{2L_0}{c} \\
    L = \frac{L_0}{\gamma} = L_0 \cdot \sqrt{1 - v^2/c^2}
\end{gather}
In order for its longitudinal rhythm to change proportionally with its slowed-down transverse rhythm, the pattern's \textbf{physical form} \textbf{must be compressed by a factor of $\gamma$} in the direction of motion. This constitutes a dynamical origin for \textbf{length contraction}.

\subsection{To Be is to be Consistent: The Physical Picture of a ``Survival Manual''}
This principle of ``To Be is to be Consistent'' is vividly illustrated in two familiar physical scenarios:
\begin{description}
\item[Scenario One: Existence Maintained through Orderly Adaptation]
Consider an electron being gently accelerated by an electric field in a large linear accelerator. The electron, this ``self-consistent pattern,'' has sufficient time and stable external conditions to \textbf{calmly and incrementally} make its internal dynamical adjustments to maintain its ``identity.'' The reason an electron remains ``an electron'' after acceleration is precisely because it has perfectly executed the dynamical scheme described by the Lorentz transformation, which preserves its own ``internal harmonic structure.''
\item[Scenario Two: Catastrophic Failure and the Reconstruction of Existence]
Now, consider a proton-proton collision in the Large Hadron Collider. This is a \textbf{catastrophic, non-adiabatic causal discoordination event}. At the moment of collision, the internal state of the newly formed, transient ``fused entity'' is extremely inconsistent. At this point, the underlying dynamics of the universe will render a judgment:
\begin{enumerate}
\item \textbf{The overwhelming outcome is ``disintegration'':} Because this fused entity cannot find any stable mode of existence, its immense energy and complex structure will rapidly ``dissolve'' into a multitude of simpler, more stable basic patterns. This vividly demonstrates the principle that \textbf{``to be inconsistent is not to be.''}
\item \textbf{The extremely rare outcome is ``creation'':} If the energy and geometric configuration of the collision ``happen'' to meet the ``nucleation conditions'' for a potential, higher-mass stable pattern, then the underlying dynamics might ``assemble,'' from that chaotic soup of energy, a brand new stable pattern that satisfies a higher level of self-consistency requirements.
\end{enumerate}
\end{description}
These two examples together paint a profound dynamical picture: the Lorentz transformation is not a ``universal straitjacket'' imposed on all objects, but rather something like a \textbf{``survival manual.''} An entity follows it because it is the only way it can continue to exist in a universe governed by a universal causal law.

\section{Derivation II: Corroboration from Soliton Dynamics—Internal Unity Manifested in a Mathematical Model}

Having completed the geometric derivation based on abstract principles, we turn to a concrete mathematical-physical model—soliton dynamics—to provide \textbf{philosophical Analogical Corroboration}. The purpose of this section is \textbf{not} to attempt to ``prove'' physical reality from a mathematical equation—because soliton-generating equations like the sine-Gordon equation are themselves Lorentz covariant, which would constitute a circular argument. Rather, the aim is to \textbf{instantiate} our aforementioned abstract philosophical concepts.

The soliton model vividly answers a key question: \textbf{How} can an extended entity, following underlying dynamical laws, spontaneously and automatically achieve the complex adjustments required to maintain ``internal unity''? It shows that seemingly strange behaviors like time dilation and length contraction can be naturally produced by a unified, non-linear dynamical process. It connects our philosophical speculations about 'self-consistency' with a computable mathematical object with a clear mechanism. Therefore, the role of soliton dynamics here is to \textbf{provide an ``existence proof'' for the philosophical concept}—it proves that such complex, adaptive dynamical behavior is possible and indeed exists in mature physical theories \cite{Skyrme1961, DrazinJohnson1989}.

By analyzing the traveling-wave solutions of such equations, we can directly ``read off'' the precise relationship between the dynamical properties of the moving soliton and its state at rest. The transformation factor is \textbf{exactly the Lorentz factor $\gamma$}:
\begin{enumerate}
    \item \textbf{Time Dilation ($T=\gamma T_0$)}
    \item \textbf{Length Contraction ($L=L_0/\gamma$)}
    \item \textbf{Energy Increase ($E=\gamma E_0$)}
\end{enumerate}
This corroboration shows that the conclusions of our derivation, based on the principle of ``causal self-consistency,'' are perfectly reproduced analytically within a mature mathematical-physical model.

\section{Discussion: Dialogue with Contemporary Philosophical Frameworks and Philosophical Implications}

\subsection{Dialogue with the Dynamical Approach and the Common Cause Explanation}
The dynamical approach of this study is highly consistent with the program of Harvey R. Brown \cite{Brown2005}. However, this study attempts to answer a more fundamental question: \textit{Why are the dynamics of matter precisely Lorentz covariant?} To this, our proposed mechanism of a \textit{``Universal Causal Framework''} and \textit{``dynamical necessity''} may be seen as an ontological supplement to Brown's program.

Meanwhile, our explanation must also be compared with Michel Janssen's famous ``common cause explanation'' \cite{Janssen2002}. Janssen argues that Lorentz covariance is a universal phenomenon because all forms of matter are subject to the same set of fundamental dynamical laws (the common cause). Our framework attempts to provide a deeper explanation for this ``common cause'' itself. We contend that this is not a coincidence, but rather because \textit{any} pattern that wishes to exist stably in a universe governed by a universal, isotropic causal framework \textit{must} emerge with this symmetry. Any dynamical law capable of supporting stable, interacting complex structures must itself be Lorentz covariant, otherwise it could not operate in a logically self-consistent causal universe.

\textbf{To further highlight this paper's philosophical contribution, it is necessary to distinguish it from J.S. Bell's classic exposition \cite{Bell1976}.} Bell, with his characteristic clarity, vividly demonstrated through an atomic model based on electromagnetic forces how length contraction could be a dynamical process. Our geometric derivation is logically similar to Bell's thought experiment. However, its philosophical goal is entirely different. Bell's argument is at the physical level: he showed that if matter is held together by specific (electromagnetic) forces, then the Lorentz transformation is its natural behavior. Our argument is at the ontological level: we claim that \textit{regardless} of the fundamental forces holding matter together, as long as it wants to exist as a stable, self-consistent pattern in a universe governed by a universal causal law (Postulate I), it \textit{must} exhibit emergent Lorentz-covariant dynamics. Our explanation, especially the derivation based on the principle of symmetry, does not depend on any specific physical force but is rooted in a deeper logic of existence.

\subsection{Reinterpretation of the ``Constancy of the Speed of Light'' and Spacetime Ontology}
This theory derives the ``constancy of the speed of light'' as an \textit{Operational Invariance}. It can be rigorously shown that the dynamical ``distortions'' experienced by the measuring instruments of a moving observer (which, as self-consistent patterns, also transform) will precisely compensate for the ``true'' change in the speed of a photon relative to the observer, such that the final measurement result is \textit{mathematically always equal to c}. This interpretation challenges spacetime realism, suggesting that spacetime itself may not be an independent entity, but rather an \textit{emergent phenomenon defined by the dynamics of matter}.

\section{Conclusion: Implications for the Foundations of Physics and Future Outlook}

Through geometric derivation and physical analogy, this paper has argued that the core of Special Relativity—the Lorentz transformation—may be understood as the necessary dynamical consequence that emerges for stable patterns to maintain their ``internal causal self-consistency'' in a universe governed by a \textit{Universal Causal Framework}. By appealing to the principle of symmetry, we have shown the logical necessity of this process, rather than it being merely a convenient assumption.

The deeper significance of this dynamical reconstruction of SR lies in the new perspectives it may offer for thinking about the foundational problems of physics. For example, it suggests a new path of inquiry for re-examining the philosophical dilemmas in General Relativity (GR). From this framework:
\begin{itemize}
    \item A \textbf{black hole singularity} might be reconceptualized as a critical point of \textit{``emergent failure.''} Here, the gravitational field is so extreme that the conditions for any known form of matter to form a ``stable self-consistent pattern'' are destroyed, and our Postulate II concerning matter no longer applies.
    \item The \textbf{information paradox} is thus opened to the possibility of being reframed as a problem of \textit{``morphological transformation''} of causal information under extreme conditions, rather than its ``loss.'' Information may not be lost, but rather encoded into a structure of the spacetime dynamics itself that we do not yet understand.
\end{itemize}

These still-exploratory ideas point to the future potential of an ontological framework based on ``universal causality'' and ``dynamical emergence.'' Future philosophical inquiry will include a deeper exploration of the metaphysical status of the ``Universal Causal Framework'' (for example, in dialogue with philosophers like Tim Maudlin on the ontology of laws and spacetime \cite{Maudlin2007}), as well as the further philosophical challenges this emergent model poses to the nature of causality and scientific measurement.

% --- Bibliography ---
\begin{thebibliography}{99}

\bibitem{Barcelo2005}
Barceló, C., Liberati, S., \& Visser, M., ``Analogue gravity,'' \textit{Living Reviews in Relativity} \textbf{8}, 12 (2005).

\bibitem{Bell1976}
Bell, J. S., ``How to teach special relativity,'' in \textit{Speakable and Unspeakable in Quantum Mechanics}, pp. 67-80 (Cambridge University Press, 1987).

\bibitem{Brown2005}
Brown, H. R., \textit{Physical Relativity: Spacetime Structure from a Dynamical Perspective} (Oxford University Press, 2005).

\bibitem{deBroglie1930}
de Broglie, L., \textit{An Introduction to the Study of Wave Mechanics} (Methuen \& Co., London, 1930).

\bibitem{DrazinJohnson1989}
Drazin, P. G., and Johnson, R. S., \textit{Solitons: an introduction} (Cambridge University Press, Cambridge, 1989).

\bibitem{Einstein1905}
Einstein, A., ``On the Electrodynamics of Moving Bodies,'' \textit{Annalen der Physik} \textbf{17}, 891-921 (1905).

\bibitem{Janssen2002}
Janssen, M., ``Causality and the two Einstein postulates of special relativity,'' in \textit{Conceptual problems of quantum gravity}, pp. 109-139 (Birkhäuser Basel, 2002).

\bibitem{Maudlin2007}
Maudlin, T., \textit{The Metaphysics Within Physics} (Oxford University Press, 2007).

\bibitem{Poincare1905}
Poincaré, H., ``On the dynamics of the electron,'' \textit{Comptes Rendus de l'Académie des Sciences} \textbf{140}, 1504-1508 (1905).

\bibitem{Skyrme1961}
Skyrme, T. H. R., ``A nonlinear field theory,'' \textit{Proceedings of the Royal Society of London. Series A} \textbf{260}, 127-138 (1961).

\end{thebibliography}

\end{document}
